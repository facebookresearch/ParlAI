

The Wizard of Wikipedia is an open-\/domain dialogue task for training agents that can converse knowledgably about open-\/domain topics! A detailed description may be found in \href{https://arxiv.org/abs/1811.01241}{\tt Dinan et al. (I\+C\+LR 2019)}.

\subsection*{Abstract}

In open-\/domain dialogue intelligent agents should exhibit the use of knowledge, however there are few convincing demonstrations of this to date. The most popular sequence to sequence models typically \char`\"{}generate and hope\char`\"{} generic utterances that can be memorized in the weights of the model when mapping from input utterance(s) to output, rather than employing recalled knowledge as context. Use of knowledge has so far proved difficult, in part because of the lack of a supervised learning benchmark task which exhibits knowledgeable open dialogue with clear grounding. To that end we collect and release a large dataset with conversations directly grounded with knowledge retrieved from Wikipedia. We then design architectures capable of retrieving knowledge, reading and conditioning on it, and finally generating natural responses. Our best performing dialogue models are able to conduct knowledgeable discussions on open-\/domain topics as evaluated by automatic metrics and human evaluations, while our new benchmark allows for measuring further improvements in this important research direction.

\subsection*{Datasets}

You can train your own Parl\+AI agent on the Wizard of Wikipedia task with {\ttfamily -\/t wizard\+\_\+of\+\_\+wikipedia}. See the \href{http://www.parl.ai/static/docs/tutorial_quick.html}{\tt Parl\+AI quickstart for help}.

The Parl\+AI M\+Turk collection scripts are also \href{https://github.com/facebookresearch/ParlAI/tree/master/parlai/mturk/tasks/wizard_of_wikipedia}{\tt made available}, for those interested in replication, analysis, or additional data collection. The M\+Turk task for evaluating pre-\/trained models is made available in this directory.

\subsection*{Leaderboard}

\subsubsection*{Human Evaluations}

\tabulinesep=1mm
\begin{longtabu} spread 0pt [c]{*{4}{|X[-1]}|}
\hline
\rowcolor{\tableheadbgcolor}\textbf{ Model }&\textbf{ Paper }&\textbf{ Seen Rating }&\textbf{ Unseen Rating  }\\\cline{1-4}
\endfirsthead
\hline
\endfoot
\hline
\rowcolor{\tableheadbgcolor}\textbf{ Model }&\textbf{ Paper }&\textbf{ Seen Rating }&\textbf{ Unseen Rating  }\\\cline{1-4}
\endhead
Retrieval Trans Mem\+Net &\href{https://arxiv.org/abs/1811.01241}{\tt Dinan et al. (2019)} &3.\+43 &3.\+14 \\\cline{1-4}
Two-\/stage Generative Trans Mem\+Net &\href{https://arxiv.org/abs/1811.01241}{\tt Dinan et al. (2019)} &2.\+92 &2.\+93 \\\cline{1-4}
Human performance &\href{https://arxiv.org/abs/1811.01241}{\tt Dinan et al. (2019)} &4.\+13 &4.\+34 \\\cline{1-4}
\end{longtabu}
\subsubsection*{Retrieval Models}

\tabulinesep=1mm
\begin{longtabu} spread 0pt [c]{*{4}{|X[-1]}|}
\hline
\rowcolor{\tableheadbgcolor}\textbf{ Model }&\textbf{ Paper }&\textbf{ Test Seen R@1 }&\textbf{ Test Unseen R@1  }\\\cline{1-4}
\endfirsthead
\hline
\endfoot
\hline
\rowcolor{\tableheadbgcolor}\textbf{ Model }&\textbf{ Paper }&\textbf{ Test Seen R@1 }&\textbf{ Test Unseen R@1  }\\\cline{1-4}
\endhead
Transformer Mem\+Net (w/ pretraining) &\href{https://arxiv.org/abs/1811.01241}{\tt Dinan et al. (2019)} &87.\+4 &69.\+8 \\\cline{1-4}
BoW Memnet &\href{https://arxiv.org/abs/1811.01241}{\tt Dinan et al. (2019)} &71.\+3 &33.\+1 \\\cline{1-4}
IR baseline &\href{https://arxiv.org/abs/1811.01241}{\tt Dinan et al. (2019)} &17.\+8 &14.\+2 \\\cline{1-4}
Random &\href{https://arxiv.org/abs/1811.01241}{\tt Dinan et al. (2019)} &1.\+0 &1.\+0 \\\cline{1-4}
\end{longtabu}
\subsubsection*{Generative Models}

\tabulinesep=1mm
\begin{longtabu} spread 0pt [c]{*{4}{|X[-1]}|}
\hline
\rowcolor{\tableheadbgcolor}\textbf{ Model }&\textbf{ Paper }&\textbf{ Test Seen P\+PL }&\textbf{ Test Unseen P\+PL  }\\\cline{1-4}
\endfirsthead
\hline
\endfoot
\hline
\rowcolor{\tableheadbgcolor}\textbf{ Model }&\textbf{ Paper }&\textbf{ Test Seen P\+PL }&\textbf{ Test Unseen P\+PL  }\\\cline{1-4}
\endhead
End-\/to-\/end Transformer Mem\+Net &\href{https://arxiv.org/abs/1811.01241}{\tt Dinan et al. (2019)} &63.\+5 &97.\+3 \\\cline{1-4}
Two-\/\+Stage Transformer Memnet &\href{https://arxiv.org/abs/1811.01241}{\tt Dinan et al. (2019)} &46.\+5 &84.\+8 \\\cline{1-4}
Vanilla Transformer (no knowledge) &\href{https://arxiv.org/abs/1811.01241}{\tt Dinan et al. (2019)} &41.\+8 &87.\+0 \\\cline{1-4}
\end{longtabu}


\subsection*{Viewing data}

You can view the standard training set with\+: \begin{DoxyVerb}python examples/display_data.py -t wizard_of_wikipedia -dt train
\end{DoxyVerb}


The knowledge returned from a standard IR system appears in the knowledge field (but you can also use your own knowledge system, accessing Wikipedia yourself, we use the dump in \char`\"{}-\/t wikipedia\char`\"{}. The field \textquotesingle{}checked\+\_\+sentence\textquotesingle{} indicates the gold knowledge the annotator labeled.

\subsection*{Pretrained models}

\subsection*{End-\/to-\/\+End generative}

You can evaluate the pretrained End-\/to-\/end generative models via\+: \begin{DoxyVerb}python examples/eval_model.py \
    -bs 64 -t wizard_of_wikipedia:generator:random_split \
    -mf models:wizard_of_wikipedia/end2end_generator/model
\end{DoxyVerb}


This produces the following metrics\+: \begin{DoxyVerb}{'f1': 0.1717, 'ppl': 61.21, 'know_acc': 0.2201, 'know_chance': 0.02625}
\end{DoxyVerb}


This differs slightly from the results in the paper, as it is a recreation trained from scratch for public release.

You can also evaluate the model on the unseen topic split too\+: \begin{DoxyVerb}python examples/eval_model.py \
    -bs 64 -t wizard_of_wikipedia:generator:topic_split \
    -mf models:wizard_of_wikipedia/end2end_generator/model
\end{DoxyVerb}


This will produce\+: \begin{DoxyVerb}{'f1': 0.1498, 'ppl': 103.1, 'know_acc': 0.1123, 'know_chance': 0.02496}
\end{DoxyVerb}


You can also interact with the model with\+: \begin{DoxyVerb}python examples/interactive.py -m projects:wizard_of_wikipedia:interactive_end2end -t wizard_of_wikipedia
\end{DoxyVerb}


\subsection*{Retrieval Model}

You can evaluate a retrieval model on the full dialogue task by running the following script\+: \begin{DoxyVerb}python projects/wizard_of_wikipedia/scripts/eval_retrieval_model.py
\end{DoxyVerb}


You can run an interactive session with the model with\+: \begin{DoxyVerb}python examples/interactive.py -m projects:wizard_of_wikipedia:interactive_retrieval -t wizard_of_wikipedia
\end{DoxyVerb}


Check back later for more pretrained models soon!

\subsection*{Citation}

If you use the dataset or models in your own work, please cite with the following Bib\+Tex entry\+: \begin{DoxyVerb}@inproceedings{dinan2019wizard,
  author={Emily Dinan and Stephen Roller and Kurt Shuster and Angela Fan and Michael Auli and Jason Weston},
  title={{W}izard of {W}ikipedia: Knowledge-powered Conversational Agents},
  booktitle = {Proceedings of the International Conference on Learning Representations (ICLR)},
  year={2019},
}\end{DoxyVerb}
 